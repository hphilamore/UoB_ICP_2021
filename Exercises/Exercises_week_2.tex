%%%%%%%%%%%%%%%%%%%%%%%%%%%%%%%%%%%%%%%%%%%%%%
% Header
\documentclass[11pt]{report}
\usepackage[english]{babel}
\usepackage[utf8x]{inputenc}
\PassOptionsToPackage{hyphens}{url}\usepackage{hyperref}
\usepackage{graphicx}
\usepackage{fullpage}
\usepackage{nicefrac}
\usepackage[lastexercise]{exercise}
\usepackage[dvipsnames]{xcolor}
\usepackage{listings}

\setlength{\parindent}{0cm}

\renewcommand{\ExerciseHeader}{\large\textbf{\ExerciseName~\ExerciseHeaderNB} - \textbf{\ExerciseTitle}\medskip}

\renewcommand{\ExePartHeader}{\medskip\textbf{\ExePartName\ExePartHeaderNB\ExePartHeaderTitle\medskip}}

\begin{document}
%%%%%%%%%%%%%%%%%%%%%%%%%%%%%%%%%%%%%%%%%%%%%%
\title{Exercises -- Week 2: New Title}
\subsubsection*{EMAT10007 -- Introduction to Computer Programming}
\section*{\Large Part 2:  Variables \& Types}



\subsection*{\Large Part 3. Control flow}

\begin{Exercise}[title=Conditionals]
    \Question{We can use Boolean operators to write conditional statements which control the flow of our program. In the interactive shell, type:
    
    \vspace{1em}
    {\tt AgeOfWhiskers = 5}\\
    {\tt if AgeOfWhiskers <= 1.5:}\\
    \hspace*{2em} {\tt print("Whiskers is a kitten")}
    \vspace{1em}
    
    What do you think will happen?
    }
    \Question{We can also use the {\tt else} keyword which triggers only when all the above conditions are false. Use an {\tt else} statement to print the string {\tt "Whiskers is an adult cat"}?}
    \Question{Wasn't it annoying retyping everything? Move your code to a file to make changing it easier.}
    \vspace{-\baselineskip}
    \Question{Add an {\tt elif} statement so the string {\tt "Whiskers is an old cat"} prints when Whiskers is older than 12.}
    \Question{Now we might want to reuse this code to test the ages of many different cats. Generalise your code to use the variable {\tt CatAge}}.
    \Question{(*) Suppose we have three circles in the $xy$-plane. Circle $C_1$ is centred at $(0, 0)$ with radius of length 5. Circle $C_2$ is centred at $(2, 1)$ and has radius of length 2. Circle $C_3$ is centred at $(-5, 0)$ and has a radius of length 3. (See Figure \ref{fig:circles})
    
    Using conditional statements write a program which takes in the variables {\tt x} and {\tt y} and tests which circles the point $(x, y)$ is in. How can you make this as concise as possible? Are there any conditions you do not have to test?
    }
    \begin{figure}[!h]
        \centering
        \includegraphics[height=5cm]{circles.png}
        \caption{Overlapping circles $C_1$, $C_2$ and $C_3$.}
        \label{fig:circles}
    \end{figure}

    % \Question{We can make this code interactive by using the {\tt input()} function. Use the {\tt input()} function to ask the user the age of their cat.}
    % \Question{(*) Can you make a game where a user tries to predict the roll of a dice? Let them know if their guess was too low or too high, and how close it was.
    
    % \textbf{Hint:} Recall you replicated dice throws in Week 1 - Ex.\ref{Ex:Import_Modules}.}
\end{Exercise}



\subsection*{Checklist}
\begin{itemize}

	\item Check that you understand the basics: variables, different types of variables (integers, floats, complex, Booleans, strings), the different built-in operators, and how these work with both numbers and strings.
	\item Practice with using Spyder as a useful calculator with access to some powerful abilities provided in ``modules''.
    \item Check you are familiar with variable naming styles and conventions.
    \item Know how to save and open a python file script in Spyder.
    \item Learn about conditional statements, and how these are used to control the flow of the program.
\end{itemize}

\subsection*{Additional resource}
\begin{itemize}
	\item There are many online tutorials for Python 3, so if you're wondering what more you can do, please do search online or ask your TAs for pointers.
\end{itemize}

\end{document}