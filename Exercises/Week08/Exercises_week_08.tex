%%%%%%%%%%%%%%%%%%%%%%%%%%%%%%%%%%%%%%%%%%%%%%
% Header
\documentclass[11pt]{report}
\usepackage[english]{babel}
\usepackage[utf8x]{inputenc}
\PassOptionsToPackage{hyphens}{url}\usepackage{hyperref}
\usepackage{graphicx}
\usepackage{fullpage}
\usepackage{nicefrac}
\usepackage[lastexercise]{exercise}
\usepackage[dvipsnames]{xcolor}
\usepackage{amsmath}


\usepackage{minted}
\makeatletter
\AtBeginEnvironment{minted}{\dontdofcolorbox}
\def\dontdofcolorbox{\renewcommand\fcolorbox[4][]{##4}}
\makeatother


\setlength{\parindent}{0cm}

\renewcommand{\ExerciseHeader}{\large\textbf{\ExerciseName~\ExerciseHeaderNB} - \textbf{\ExerciseTitle}\medskip}

\renewcommand{\ExePartHeader}{\medskip\textbf{\ExePartName\ExePartHeaderNB\ExePartHeaderTitle\medskip}}

\begin{document}

%%%%%%%%%%%%%%%%%%%%%%%%%%%%%%%%%%%%%%%%%%%%%
\subsubsection*{Introduction to Computer Programming}
\subsection*{\Large Exercises -- Week 8. Mathematical computations with Python}

\subsection*{\Large Part 1. SymPy}

\begin{Exercise}[title=Getting familiar with SymPy (Essential)]

  \Question{In SymPy, the keywords {\tt pi} and {\tt E} define the special
    numbers $\pi$ and $e$. Use the {\tt pprint} or {\tt display} functions 
    to show that the floating-point representations
    of these numbers can be computed using {\tt pi.evalf()} and {\tt E.evalf()}.
    \textbf{Fun fact}: Typing {\tt pi.evalf(n)} computes the first $n$
    digits of $\pi$. Try executing {\tt pi.evalf(100)}.}
  
  \Question{Compute $\cos(\pi/2)$ and show that it is equal to 0.
    Compute $\log(e^2)$ and show that it is equal to 2. \textbf{Note:}
    the {\tt log} function in SymPy computes the natural logarithm.
    }

  \Question{The exponential function $e^{x}$ is defined in SymPy
    as {\tt exp(x)}. Compute {\tt exp(1)} and assign
    it to a variable {\tt my\_e}. Use the {\tt evalf} method to compute
    a floating-point approximation to {\tt my\_e}. You should find that the value
    is the same as {\tt E.evalf()}.}

  \Question{Suppose that we want to assign the variable {\tt a} the
    exact value of $1/13$.
    Try to do this by executing {\tt a = 1/13} and then printing the
    result using the {\tt pprint} or {\tt display} functions.
    You should see that a float is computed instead of an exact
    symbolic expression. To get around this issue,
    SymPy has a special variable type for fractions called Rational.
    Use the command {\tt a = Rational(1, 13)} to create a variable of
    type Rational with value $1/13$ and assign it to {\tt a}. Then print this value to
    see that the exact representation of $1/13$ has been retained.}

  \Question{Compute the exact value of
    $43/452 + 54/725 - 10/98$.}
    

\end{Exercise}


\begin{Exercise}[title=Functions (Essential)]

  \Question{Define the symbol $x$ and the function
    \begin{align}
      y(x) = \frac{x^2 - 4x + 3}{1 + x^2}
    \end{align}
    Compute the values $y(0)$ and $y(4)$ exactly. Compute
    the floating-point approximation to $y(4)$.}

  \Question{Plot $y(x)$ over the range from $x = -10$ to $x = 10$.}

  \Question{Compute the roots of the function $y$; that is, determine
    the values of $x$ such that $y(x) = 0$.}

  \Question{SymPy can also compute limits. Look up how to do this using
    {\tt help(limit)}. Compute
    \begin{align}
      \lim_{x \to \infty} y(x).
    \end{align}
  }

  \Question{Use the {\tt diff} function to compute the derivative of $y(x)$,
    which we denote by $y'(x)$. Compute $y'(0)$.}

  \Question{Compute the integral given by
    \begin{align}
      A = \int_{0}^{1} y(x)\, dx.
    \end{align}
    Then, determine a floating-point approximation to $A$.}

  \Question{(Advanced) Determine the maximum and minimum values of the
    function $y$. \textbf{Hint}: The positions of the maximum and minimum
    can be determined by solving $y'(x) = 0$. Then evaluate $y(x)$ at these
    points. Show that these points are maxima and minima by computing
    $y''(x)$ at these points and checking the sign.}
  

\end{Exercise}

\begin{Exercise}[title = Modelling a pandemic (Essential)]

  The increase in the number of individuals infected with a virus can
  be described by the logistic function
  \begin{align}
    P(t) = \frac{K P_0}{P_0 + (K - P_0)e^{-r t}}.
  \end{align}
  In this equation, $P$ is the number of infected individuals,
  $t$ is the number of days since the outbreak of infection, $P_0$ is the
  number of infected individuals at the time of outbreak,
  $r$ is the rate of infection, and $K$ is the maximum number of
  individuals who can be infected.

  \Question{Create symbols {\tt t, r, K, P\_0} for the variables that
    appear in the logistic function. Store the function $P(t)$ in a
    variable called {\tt P}.}

  \Question{Create a dictionary called {\tt params} that stores
    the parameter values $K = 67$, $r = 0.1$, and $P_0 = 0.5$.
    \textbf{Hint}: treat the symbols as keys when creating the
    dictionary.}

  \Question{Substitute the dictionary of parameter values into the function
    $P(t)$ and plot it over the range from
    $t = 0$ to $t = 100$,
    You should see that after the outbreak
    there is a period of rapid, exponential growth. Then, the growth
    of the infected population begins to slow because there is no one left
    to infect.}

  
  \Question{Use the {\tt diff} function to compute the derivative of $P$,
    which we'll denote by $P'(t)$. Substitute the parameter values
    into $P'(t)$ and plot it from $t = 0$ to $t = 100$. (Advanced)
    Estimate when the population of
    infected individuals is increasing the fastest.}

  \Question{Using the {\tt solve} function, obtain an equation for the
    time at which half of the population has been infected by the
    virus. \textbf{Hint}: Half of the total population is $K/2$ so
    the equation we want to solve is $P(t) = K/2$, which can be
    reformulated as $P(t) - K/2 = 0$.}

  \Question{Substitute the parameter values into your expression
    to show that it takes about 50 days (7 weeks) for half of the
    population to be infected by the virus.
    Then, create a second parameter set where the infection rate is
    half of the original, so $r = 0.05$. Show that this
    \emph{doubles} the time it takes for half of the population to
    become infected. Hence, reducing the infection rate can
    substantially delay the spread of the virus.}

  \end{Exercise}


  \begin{Exercise}[title=The roots of unity (Advanced)]

	The $n$-th roots of unity are the solutions $z$ of the equation
    $z^n = 1$. The fundamantal theorem of algebra tells us that
	there must be $n$ solutions to this equation.  This question
	will look at computing these solutions.

  \Question{Use SymPy to compute the tenth roots of unity by setting
    $n = 10$ and then solving for $z$.}
  \Question{Use a list comprehension along with the
    {\tt evalf} method to convert the
    list of exact values that SymPy computes into a list of floats.}
  \Question{Use another list comprehension to compute floating-point approximations to
    the numbers $z = e^{2 \pi i m / 10}$ where $i = \sqrt{-1}$ is the
	imaginary number and $m = 0, 1, \ldots, 9$.  You should see that
	these are the exactly the roots of unity computed by
    SymPy (but they are ordered differently in the lists).}
  \Question{Explore this problem using different values of $n$. Can you think of a formula for the $n$-th roots of unity?}

\end{Exercise}


 
\begin{Exercise}[title=The period of a pendulum (Advanced)]

  Consider a pendulum that is made up of a sphere of mass $m = 1$~kg that is attached to a string of length $L = 0.5$~m.  Suppose that the mass is displaced so that it makes an angle $\theta_0$ with respect to the vertical.
When the mass is released, it will swing back and forth.  The purpose of this question is to calculate the period of the oscillations, which is the time
required for the pendulum to swing one full cycle, that is, to return to
its initial position once it has been released.

\Question{If the initial angle $\theta_0$ is small, then the pendulum undergoes simple harmonic motion.  The
period of oscillations, denoted by $T$, is given by the equation
\begin{align}
T = 2 \pi \sqrt{\frac{L}{g}},
\label{eqn:T_lin}
\end{align}
where $g = 9.8$~m/s$^2$ is the gravitational acceleration. 
Using Equation \eqref{eqn:T_lin}, compute the period
of oscillations with provided parameter values. \textbf{Hint:}
use the function {\tt sqrt()} to compute the square root.}

\Question{The general expression for the period of a pendulum, which is valid for any value of $\theta_0$, 
is given by the integral
\begin{align}
T = 2 \sqrt{\frac{L}{g}} \int_{-\theta_0}^{\theta_0} \frac{d \theta}{\sqrt{2(\cos \theta - \cos \theta_0)}}.
\end{align}
By taking the initial angle to be $\theta_0 = \pi / 4$ (this is equivalent to
$45^\circ$) and using the values of the length $L$ and gravitational
acceleration $g$, compute a floating-point approximation to this
integral. \textbf{Note}: this integral cannot be evaluated exactly,
so a floating-point approximation is the best we can do!}

\Question{Using a loop, compute the period when
  $\theta_0 = 0.1 \pi$, $0.2 \pi$, ...,
  $0.9 \pi$ and print the result at each iteration.
  What happens to the period $T$ as $\theta_0$ increases?
    \textbf{Note:} these calculations may take
  a few minutes depending on the speed of the computer you are using.}

\end{Exercise}


\subsection*{\Large Part 2. NumPy}

\begin{Exercise}[title=Creating arrays and accessing elements (Essential)]

  \Question{Create a variable $a$ that stores the array
    [5, 4, 9, 2, 0, 4, 7, 2].}

  \Question{Print the first entry of $a$ and the last entry of $a$
    (\textbf{Hint}: You can use the index $-1$ to access the last entry
    of an array). The colon operator $:$ can be used to access
    sequential elements of an array. Print the values of
    {\tt a[1:6]} and explain the output.}

  \Question{Change the last entry of $a$ to $-9$ and print the result.
    Now run the command {\tt a[0:3] = 1} and print the result. How has this
    altered the array $a$?}

  \Question{Create an array $r$ that contains 20 random integers between
    $1$ and $9$ (inclusive). Print the result. This array will be used in the
    next question.}

  \Question{(Advanced) \textbf{Logical indexing} provides a quick way to access and modify
    entries in an array that satisfy certain criteria. In this question, we'll use
    logical indexing to replace all of the entries in $r$ that
    are smaller than 5 with 0. First, run the command {\tt idx = r < 5}.
    Print the value of {\tt idx}. Explain the result you see. 
    Now run the command {\tt r[idx] = 0} and print the value of $r$.
    What has happened?}

  \Question{Create a variable $A$ to store the matrix
    \begin{align}
    \begin{pmatrix}
      6 & 2 & 3 \\ 4 & 4 & 1 \\ 8 & 5 & 6
    \end{pmatrix}
    \end{align}
    as a NumPy array.
  }

  \Question{Change the entry in the second row, first column of $A$ to 9. Then
    change the entry in the last row and last column of $A$ to 0. Print the
    updated array $A$. The $n$-th row of $A$ can be accessed using the colon
    operator as {\tt A[n-1,:]}. Similarly, the $m$-th column of $A$ can be
    accessed using {\tt A[:,m-1]}. Use the colon operator to print the entries
    in the second row of $A$.}

  \Question{Create a $2 \times 2$ array of zeros and assign it to a variable $A$. 
    Use the colon
    operator to set the first row of $A$ to 1 and the second row of $A$ to 2.
    \textbf{Hint}: The operation {\tt A[n-1,:]~=~q} will set all of the entries in
    the $n$-th row of $A$ to the value $q$.
    Using a {\tt for} loop, create a $5 \times 5$ matrix where the entries in
    row $i$ are equal to $i$.}
  
\end{Exercise}


\begin{Exercise}[title=Performing operations on NumPy arrays (Essential)]

  \Question{Create two NumPy arrays to store the vectors
    $a = (3, 5, 2)$ and $b = (6, 3, 1)$. Calculate
    $c = a + 2b$. Calculate the dot product $a \cdot b$ using
    the {\tt dot} method or the {\tt dot} function.
    Can you also compute the dot product using
    element-by-element multiplication along with the {\tt np.sum}
    function? Recall that the dot product
    is defined as $a\cdot b = \sum_{i} a_i b_i$.}

  \Question{Create an array called $t$ that contains 500 values between
    0 and 5. Create a
    second array called $y$ that stores the values of $y = t^2 e^{-2 t}$.
    \textbf{Hint}: use the {\tt exp} function to compute the exponential
    of a NumPy array. Find the maximum value of $y$. \textbf{Note}: this is
    a simple way of finding the maximum of a function. 
    (Advanced): Use logical indexing or otherwise to find the value of $t$ at
    which $y$ is maximal.
    (Optional but helpful):
    Use SymPy to plot the function $y(t)$ and check that your answers
    are correct.}

  \Question{This question will demonstrate that NumPy can also integrate
    functions. Unlike SymPy, NumPy can only produce approximate values of
    integrals. Create a NumPy array $x$ that stores 50 values between 0 and 5.
    Create the array $y = x / (x + 1)$. Look up how to use NumPy's {\tt trapz}
    function, which uses the trapezoid rule to approximate integrals.
    Use {\tt trapz} to compute $\int_{0}^{5} y(x)\, dx$. Use SymPy to compute
    this integral exactly and compare with the NumPy result. What happens
    if you repeat the calculation using 500 points between $0$ and $5$?}

  \Question{The table below provides the gravitational acceleration $g$ of each
    of the planets:
    \begin{table}[hp!]
      \centering
      \begin{tabular}{c|c}
        Planet & $g$ [m/s$^2$] \\ \hline
        Mercury & 3.7 \\
        Venus & 8.9 \\
        Earth & 9.8 \\
        Mars & 3.7 \\
        Jupiter & 25\\
        Saturn & 10\\
        Uranus & 8.9\\
        Neptune & 11\\
      \end{tabular}
    \end{table}
    \\
    Use NumPy functions to compute the maximum, minimum, mean, and median values of $g$.}

  \Question{Create NumPy arrays to store the matrices
    \begin{align}
      A = \begin{pmatrix}
        1 & 2 & 3 \\ 3 & 2 & 1 \\ 2 & 4 & 6
      \end{pmatrix},
                                          \quad
                                          B = \begin{pmatrix}
                                            1 & 5 & 0 \\ 0 & 1 & 1 \\ 4 & 3 & 1
                                          \end{pmatrix}
    \end{align}
    Calculate $C = A + 2B$. Then compute $AB$ and $BA$. You should notice that
    $AB \neq BA$.
  }

  \Question{A common operation to perform on matrices is to turn the rows into
    columns and the columns into rows. This is called transposing the matrix.
    Use the function {\tt transpose} to compute the transpose of $A$ and
    print the result.}

  \Question{Solve the linear system of equations $Ax = b$ where
    \begin{align}
      A = \begin{pmatrix}
        1 & 0 & 0 & -1 \\
        1 & -2 & 1 & 0 \\
        0 & 1 & -2 & 1 \\
        2 & 0 & 0 & 1
      \end{pmatrix},
                    \quad
                    b = \begin{pmatrix}
                      0 \\ 1 \\ 1 \\ 2
                    \end{pmatrix}
    \end{align}
    Print the array $x$. Then compute $Ax - b$ and print the result.
  }
  
\end{Exercise}


\begin{Exercise}[title=Weather prediction (Advanced)]

  In this example we'll use a \textbf{Markov chain} to create a simple
  model for weather prediction. To start, we will assume that there are
  three states of weather: sunny, cloudy, and rainy. We
  will use the state vector $x = (x_0, x_1, x_2)$ to describe the probabilities
  of the weather being sunny ($x_0$), cloudy ($x_1$), or rainy ($x_2$).
  We use a transition matrix $P$ to describe how the weather changes from one
  day to the next. The entries of the transition matrix, $P_{i,j}$, describe
  the probability of going from state $i$ to state $j$. For this problem,
  we will assume that
  \begin{align}
    P = \begin{pmatrix}
      0.5 & 0.3 & 0.2 \\
      0.4 & 0.2 & 0.4 \\
      0.6 & 0.2 & 0.2
    \end{pmatrix}
  \end{align}
  The entry $P_{1,1} = 0.5$ means there is a 50\% chance that if a day is sunny,
  then the next day will be sunny. Similarly, $P_{3,1} = 0.6$ means there is a
  60\% chance that if a day is rainy, then the next day will be sunny.
  
  \Question{Suppose that today is sunny. Then we can write the state vector
    as $x^{(0)} = (1, 0, 0)$. The weather tomorrow can be predicted by
    computing
    the product $x^{(1)} =  x^{(0)} P$.
    What is the probability that tomorrow will be sunny?}

  \Question{The product $x^{(2)} = x^{(1)} P = x^{(0)} P^2$
    can be used to predict the weather
    in two days. What is the probability that it will rain in two days?}

  \Question{Provide a prediction of the weather for seven days from now. That is,
    compute $x^{(7)}$.}

\end{Exercise}


\end{document}