%%%%%%%%%%%%%%%%%%%%%%%%%%%%%%%%%%%%%%%%%%%%%%
% Header
\documentclass[11pt]{report}
\usepackage[english]{babel}
\usepackage[utf8x]{inputenc}
\usepackage{hyperref}
\usepackage{graphicx}
\usepackage{fullpage}
\usepackage{enumitem}
\usepackage{color,soul}
\usepackage[lastexercise]{exercise}

\setlength{\parindent}{0cm}

\renewcommand{\ExerciseName}{Part}

\renewcommand{\ExerciseHeader}{\large\textbf{\ExerciseName~\ExerciseHeaderNB} - \textbf{\ExerciseTitle}\medskip}

\renewcommand{\ExePartHeader}{\medskip\textbf{\ExePartName\ExePartHeaderNB\ExePartHeaderTitle\medskip}}

\begin{document}
%%%%%%%%%%%%%%%%%%%%%%%%%%%%%%%%%%%%%%%%%%%%%%
\title{Assignment 1: Small Project - Building Programs}
\subsubsection*{EMAT10007 -- Introduction to Computer Programming}
\subsection*{\Large Assignment 2021 -- Encrypted Information}

\section*{Overview}
\begin{itemize}
	\item The objective of the assignment is to submit an interactive Python program that allows the user to:
	\begin{itemize}
		\item Input a message to be encrypted/decrypted using a Caesar cipher.
    	\item Specify the rotation used by the Caesar cipher.
    	\item Automate the identification of the rotation needed to decipher an encrypted message.
		\item Extract statistics about the messages, such as letter frequency, average word length, etc.
	\end{itemize}
% 		\item To automate the identification of the rotation used by some encrypted message.
	
	\item Your Python program should be accompanied by a {\bf short, 1-2 page} report, submitted as a .pdf file called report.pdf. This should discuss your implementation and any design choices. Your report should contain the following sections:
	\begin{itemize}
		\item Introduction: An overview/summary of what your program does to show that you have attempted to implement all the required components.
    	\item Analysis: Discuss what programming techniques you chose to meet the objectives of the assignment and discuss why you made these decisions. For example, why you chose to use a specific data structure or how you improved the re-usability of your code. 
    	\item Conclusion: Discuss anything in your program that you think could be improved. If possible, make
suggestions for what you could try to do to improve it if you had more time.
	\end{itemize}
    You must reference any external code or ideas used, including code snippets you may have found online.
	
	
 	\item \textbf{The \textbf{deadline is 13:00 on Friday 10th Decemeber}} You must upload your assignment including the program (.py), report (.pdf) and any output files generated by the program (.csv, .pdf) to \textbf{Blackboard}.
	\item 65 marks available:
	\begin{itemize}
		\item Exercise 1-5 [50 marks]
    	\item Report [10 marks]
    	\item \textbf{readability} and \textbf{re-usability} of your code (sensible naming, use of comments and documentation strings, tidy code, avoiding repetition). [5 marks]
	\end{itemize}
	\item You must show which part of the assignment each section of your code answers by adding comments showing part and sub-section (e.g. Part 1.1) using the following format:

    \vspace{0.5em}
    {\tt \#\ PART 1.1 Comment here ...}
    \vspace{0.5em}
    
    Note that the order that the answers to the questions appear in the code may be different to the order they appear in this document so it is important to indicate to the marker where you have attempted to answer each question. You should also use comments to show what your code does. 
    \item This is an individual project. You may discuss the creative approaches to solve your assignment with each other, but you must work individually on all of the programming. 
    	\textbf{We will be checking for plagiarism!}
    \end{itemize}
    
\vspace{1em}
\hrule
\vspace{1em}
    
    \subsection*{Helpful hints and suggestions}
\begin{itemize}
    \item Complete all of the exercise sheets first if you haven't already - they have been designed to prepare you for this assignment.
	\item Plan the flow of your program \textbf{before} you start programming. Look back at previous exercise sheets to see how to approach larger problems by breaking them down into smaller ones.
	
% 	\item \textbf{Comment your code!} Comments are crucial for many reasons:
% 	\begin{itemize}
% 		\item They help other readers to understand what your code is doing.
% 		\item They help \emph{you} remember what your code is doing, when you come back to it weeks/months/years later. The whole point of writing code is that it can be reused many times, so to make it reusable, add comments.
% 		\item To help your TA assess your understanding of the problem you are trying to solve, and to assess how well your solution solves it. If you use a data structure to implement some functionality, explain it in the comments.
% 	\end{itemize}
% 	Short comments are great to summarise the next several lines of code by giving a high-level overview. For longer, more detailed comments (likely useful in functions and for Part 4), by enclosing them in three speech marks {\tt """} either side. E.g.
	
% 	\vspace{0.5em}
% 	{\tt def SomeFunction(SomeArgument):}\\
% 	{\tt \hspace*{2em}"""}\\
% 	{\tt \hspace*{2em}This function does ...}\\
% 	{\tt \hspace*{2em}....and it does this....}\\
% 	{\tt \hspace*{2em}"""}
% 	\vspace{0.5em}
	
% 	\item You will gain marks for: % mark for the following points:
% 	\begin{itemize}
% 	    %\item Functionality: Does your program do what it is supposed to. %The more of the functionality described above you implement, the higher the mark.
% 	    \item Types: Proper use of data types and data structures. 
% 	    \item Comments: Appropriate and useful comments. 
% 	    \item Naming: Appropriate variable and function names 
% 	    \item Report: Informative report, which explains your thought process and analyses the choices you made in your code design. You must reference any external code or ideas used.
% 	    \item Working code: Does the code do what it is supposed to? {\bf READ THE QUESTIONS CAREFULLY to check}. Is the program robust, i.e., does it deal correctly with wrong inputs (user/program interaction).
% 	\end{itemize}
	
	\item Finally, don't forget to ask for help! You will be able to ask lecturers and TAs for help and advice during the weekly drop-in sessions. 
\end{itemize}

    
\newpage

\section*{Background: The basics of cryptography}
\begin{itemize}
	\item A Caesar cipher is a simple, well known cipher used in the encryption of strings. It is a `substitution cipher', meaning that each letter is \emph{substituted} with a corresponding letter in the alphabet at a predefined offset from the input letter's position. The the size of the offset is called the \emph{rotation}.
	\item The table below shows a Caesar cipher with a rotation value of $13$; a popular special case of the Caesar cipher known as \href{https://www.wikiwand.com/en/ROT13}{ROT13}. 
	\begin{table}[h]
		\centering
		\begin{tabular}{|r|c|c|c|c|c|c|c|c|c|c|c|c|c|c|c|c|}
		\hline
		\textbf{Index} & {\tt 0} & {\tt 1} & {\tt 2} & {\tt 3} & {\tt 4} & {\tt 5} & {\tt 6} & {\tt 7} & {\tt 8} & {\tt 9} & {\tt 10} & {\tt 11} & {\tt 12} & 13 & ... \\ \hline
		\textbf{Input} & A & B & C & D & E & F & G & H & I & J & K  & L  & M & N & ... \\ \hline
		\textbf{Output} & N & O & P & Q & R & S & T & U & V & W & X  & Y  & Z & A & ... \\ \hline
		\end{tabular}
	\end{table}
	\item In this example, the letter ``A'' is replaced by the letter indexed $13$ positions to the right of ``A''; the letter ``N''. \\ Because there are 26 letters in the alphabet, ROT13 replaces each letter by its partner 13 characters further along the alphabet. For example, HELLO becomes URYYB (or, conversely, URYYB becomes HELLO again).
	\item The Caesar cipher is only defined for the letters of the alphabet, meaning that punctuation, spaces, and numbers are left unchanged.
	\item The Wikipedia page contains more information
	about the Caesar cipher: \\ \url{https://en.wikipedia.org/wiki/Caesar_cipher}
\end{itemize}

\subsubsection*{Caeser Cipher Examples}

Input: {\bf hello world} \\
Encrypt, rotation 13 \\
Output: {\bf uryyb jbeyq} \\

Input: {\bf hello world} \\
Encrypt, rotation 3 \\
Output: {\bf khoor zruog} \\

Input: {\bf khoor zruog} \\
Decrypt, rotation 3 \\
Output: {\bf hello world} \\

Input: {\bf hello world} \\
Encrypt, rotation -3 \\
Output: {\bf ebiil tloia} \\

\newpage

\begin{Exercise}[title=Encryption and Decryption]\label{Ex:Encryption_Decryption}

{\bf Total: 13 marks}
% 	\Question{The file that is run to execute your program should be named {\bf FirstnameSurname.py}, for example, IsaacNewton.py or AdaLovelace.py
% 	}

    \Question{The file that is run to execute your program should be named {\bf main.py}
    }
	\Question{When run, the program should ask the user for:
	\begin{itemize}
		\item The {\bf cipher mode} : User should enter {\tt E} to encrypt a message or {\tt D} the decrypt a message
		\item A {\bf rotation mode} : User should enter {\tt manual} to choose a numerical value or {\tt random} for the program to select a random integer value. 
		\begin{itemize}
		    \item If {\tt manual} is selected, the user should be prompted to enter an integer value indicating how many places the cipher should shift each character and this number should be assigned to a variable {\tt rotation} as a numerical value.
		    \item If {\tt random} is selected a random integer value should be generated and assigned to a variable {\tt rotation} as a numerical value.
		\end{itemize}
		\item A {\bf message} : User should enter a message to be encrypted or decrypted. 
	\end{itemize}
	{\bf[3 marks]}
	}
% 	The only actions required of the user to operate the cipher should be to:
% 	\begin{enumerate}[label=(\roman*)]
%         \item run the python program
%         \item provide these three inputs when prompted
%     \end{enumerate}
	\Question{If no input is given, or incorrect input is given (e.g. no rotation value is entered, or the user enters a value other than {\tt E} or {\tt D} when asked if they want to encrypt or decrypt a message), the program should print an error and prompt the user for the input again. \newline {\bf[1 mark]} }
	\Question{When the user selects encryption mode by entering {\tt E}, the program should call a function called {\tt encrypt}. {\tt encrypt} should take two inputs, the rotation value and message and return the encrypted message as shown in the Caeser cipher examples on the previous page. 
	\newline {\bf[3 marks]} }
	\Question{When the user selects decrpytion mode by entering {\tt D}, the program should call a function called {\tt decrypt}. {\tt decrypt} should take two inputs, the rotation value and message and return the decrypted message (Note that decryption follows the same process as encryption, only the shift goes the opposite way) as shown in the Caeser cipher examples on the previous page. 
	\newline {\bf[3 marks]} }
	\Question{The program should print the encrypted message if the cipher mode is {\tt encrypt} and the decrypted message if the cipher mode is {\tt decrypt}. \newline The message should be printed in \textbf{UPPER CASE} letters. \newline Numbers, punctuation and spaces should be unchanged. \newline {\bf[3 marks]} }
	
\end{Exercise}



% \begin{Exercise}[title=Analysing messages]

% The program should include a function called analyse\_msg.
% \Question{You may try to improve on the automated decryption process the following ways:

% \end{Exercise}

\begin{Exercise}[title=Analysing Messages]\label{Ex:Analysing_Msgs}

{\bf Total: 12 marks}

\ExeText{In this exercise, the defnition of a word is a collection of characters with a space at the
leading and trailing end of the characters, excluding any numbers and punctuation marks: 
\begin{itemize}
        \item python is a word 
		\item 123 is not a word 
		\item py8thon should be interpreted as the word python
		\item python! should be interpreted as the word python
\end{itemize}

The program should include a function called {\tt analyse\_msg}. \\ {\tt analyse\_msg} must do the following:}
    \Question{Take one input argument, the plaintext ({\bf un-encrypted}) message (i.e. whether {\tt encrypt} or {\tt decrypt} is used, the un-encrypted message should be input to  {\tt analyse\_msg}). 
    \newline {\bf[1 mark]}} 
    \Question{Compute the following data on the input:
	\begin{enumerate}[label=(\alph*)]
		\item Total number of words \newline {\bf[1 mark]}
		\item Number of unique words \newline {\bf[1 mark]}
		\item Minimum word length \newline {\bf[1 mark]}
		\item Maximum word length \newline {\bf[1 mark]}
		\item Most common letter \newline {\bf[1 mark]}
	\end{enumerate}}
	\Question{Save each metric in a .txt file called metrics.txt in the same directory as the program. Each metric:
    \begin{itemize}
        \item total number of words 
		\item number of unique words 
		\item minimum word length
		\item maximum word length
    \end{itemize}
    should appear on a new line with the following format:
    \begin{verbatim} total number of words: 57 \end{verbatim}
    
    (i.e. total number of words is 57)
    
    {\bf[3 marks]}
    }
    
    % \Question{Compute (up to) the ten most common words sorted in descending order by the number of times they appear in the message. \\ If there are ten or fewer words in total in the message, sort all words in the message. \\ If there are more than ten words in total, select only the ten most common words. \\ If there are multiple words with the same frequency, such that including all these words would mean that the sorted list of words exceeds 10, then exclude these words, so that the list length does not exceed 10. \newline {\bf[2 marks]}}
    
    
    \Question{Sort all unique words by the frequency that they appear in the message, in descending order, and print (up to) the ten most frequently occurring words.  \\
    If there are ten or fewer unique words in the message, include all unique words. \\ If there are more than ten unique words, include only the ten most common words. \\ If there are more than ten unique words, and multiple unique words with the same frequency, such that the sorted list of unique words exceeds length 10, then exclude these words, so that the list length does not exceed 10. \newline {\bf[2 marks]}}
    
	\Question{Print the (up to) ten most common words sorted in descending order: with the following format:
	
	\vspace{0.5em}
    {\tt the:~4} 
    \vspace{0.5em}
    
    
    (i.e. `the' has been found 4 times)
    \newline {\bf[1 mark]}}
    
\end{Exercise}

%\newpage
% \newline
% \newline

\begin{Exercise}[title=Messages from a file]

{\bf Total: 6 marks}

    \Question{Modify your program so that {\bf before} prompting the user to enter the message to encrypt/decrypt the user is prompted to select a {\bf message entry mode} by entering:
    \begin{enumerate}[label=(\alph*)]
		\item {\tt manual} : to indicate the user will type in a message to encrypt/decrypt
		\item {\tt file} : to indicate the user will enter the name of a text file, the contents of which will be encrypted/decrypted
	\end{enumerate} 
	{\bf[1 mark]}}
	
    \Question{If the user enters {\tt manual} when prompted, the user should be asked to enter the message directly, as in Part \ref{Ex:Encryption_Decryption}. If the user instead chooses {\tt file}, the user should alternatively be asked to enter a filename (or file path if the file is located in another directory). \newline {\bf [2 marks]}}
    \Question{In either case, if no input is given, or incorrect input is given (e.g. if no message is entered is given, or the filename provided cannot be found, the program should print an error and prompt the user for the input again. The program should then continue to work as before.\newline {\bf [2 marks]} }
    \Question{If the user chooses {\tt file}, the program should read the contents of the file into a variable
    that is then passed as an argument to the {\tt encrypt}/{\tt decrypt} functions.\newline {\bf[1 mark]}}
\end{Exercise}

\begin{Exercise}[title=Automated decryption]
	
	{\bf Total: 11 marks}
	\Question{Modify your program so that when run, the program will accept a new option when it prompts the user to enter the cipher mode: The user should enter {\tt E} to encrypt a message or {\tt D} the decrypt a message as in Part \ref{Ex:Encryption_Decryption}), but should now additionally accept {\tt A} indicating that the user wants to auto-decrypt the message.\newline {\bf[1 mark]}}
	\Question{If {\tt A} is entered, the program should call a function called {\tt auto\_decrypt}. \\ {\tt auto\_decrypt} should take two inputs, the rotation value and message. \newline {\bf[1 mark]}}
	\Question{The program should then automate the decryption process by implementing the following algorithm:
	\begin{enumerate}[label=(\alph*)]
	    \item Read in {\tt words.txt}, a file of common English words (this is provided and can be downloaded from BlackBoard).
		\item Iterate through all possible rotations, applying the rotation to the first 10 words only (or all words if the message contains 10 words or less). %and apply the decryption function to the first line of the message.
		\item During each iteration:
		\begin{enumerate}[label=(\roman*)]
		    \item attempt to match words in the rotated first line with words found in the common words list.
		    \item If one or more matches are discovered, then present the line to the reader, and ask if the line has been successfully decrypted:
		    %\begin{itemize}
		        \item If the user answers ``no'', then continue to iterate until the first line is successfully decrypted.\\
		        %\item 
		        If the user answers ``yes'', then apply the successful rotation to decrypt the rest of the file.
		    %\end{itemize}
		    \item Return the decrypted message. \\ In the case that no successful decryption is found, the program will be unable to collect the metrics (Part \ref{Ex:Analysing_Msgs}) on the plaintext message. You should account for this in your code and ensure that, if no match is found, the program skips collecting the metrics. 
		\end{enumerate}
	\end{enumerate}
    {\bf[9 marks]}
    }
\end{Exercise}

% \begin{Exercise}[title=Using Data From imported files]
%     \Question{Use your program to decrypt the file {\tt douglas\_data\_encrypted.txt}, a data set about wooden beams.}
%     \Question{Plot the {\bf knot ratio}  against the {\bf bending strength}  of the beams. \newline Fit a trend-line to the data and display this on the same plot along with the equation of the fitted line.}
%     \Question{The root mean square error (RMSE) is metric often used to judge how well a trend line fits the data. The RMSE is the square root of the mean of the sum of the error, $\varepsilon_i$, squared, for all data points. A smaller RMSE indicates a better fit between raw and fitted data.  $$RMSE=\sqrt{\frac{1}{N}\sum_{i=1}^{N}{\varepsilon_i^2}}$$
%     The error $\varepsilon_i$ is the difference between the raw data point, $a$, and the fitted data point, $y$, for some input value $x_i$:
% $$\varepsilon_i = a(x_i) - y(x_i)$$ Find the RMSE for the raw data and fitted data on the knot ratio vs. beam strength plot and display the value on the plot.}
%     \Question{Save the plot as a .pdf file.}
%     \Question{The bending stiffness $k$ of a cantilever beam can be found by: $$ k = \frac{3EI}{L^3} $$ where $E$ is the Young's modulus, $L$ is the distance from the beam support, $I = \frac{b^4}{12}$ is the area moment of inertia of a beam with a square cross section with sides of length $b$ (Figure 1).%} 
    
%     \vspace{5mm}
    
%     % \begin{figure}[t]
%     % \begin{center}
%     % \includegraphics[width=7cm]{cantilever.png}
%     % \caption{Parameters of a square beam configured as a cantilever}
%     % \label{cantilever}
%     % \end{center}
%     % \end{figure}
%     %}
    
%     %\makebox[0pt][l]{%
%     \begin{minipage}{\textwidth}
%     %\centering
%     \begin{center}
%     \includegraphics[width=.5\textwidth]{cantilever.png}
%     %\caption{Figure 1: }{figure caption}
%     \label{fig:fig1}
%     \end{center}
%     \end{minipage}
    
%     \vspace{5mm}
    
%     \begin{center}
%     \caption{Figure 1: Beam configured as a cantilever showing parameters to calculate bending stiffness}
%     \end{center}
    
%     \vspace{5mm}
%     }
    
    
%     Modify your program to:
%     \begin{enumerate}[label=(\alph*)]
% 		\item Use parameters {\tt E} and {\tt BEAMHEIGHT} from the decrypted file to find the bending stiffness in Nmm$^{-1}$ for each beam. Assume the beam is configured as a cantilever where $L = $ ({\tt beam height} - $d$), $d = 10$ cm. {\bf Take care to check the units used in the decrypted file!}
% 		\item Save the data-set as a .csv file:
% 		\begin{itemize}
% 		    \item Include the bending stiffness as an extra column.
% 		    \item Exclude the {\tt SAMPLE} column from the original data-set.
% 		\end{itemize} 
% 	\end{enumerate}
%     }

% \end{Exercise}

% % \begin{Exercise}[title=Importing mulitple files]
% % 	\Question{Modify your program so the user now has an additional choice of importing a directory of files in addition to providing an input message and reading a message from a file. Like when importing a file, the program should first check if the directory exists and raise an error of it is not found that prompts the user to enter the name correctly.}
% % 	\Question{If the option to import directory is chosen, the program should read in directory, and decrypt/encrypt the contents of each file.}
% % 	\Question{Modify the auto-decryption function so that the input from the user to verify if the decryption was successful can be switched on/off using an input argument to the auto-decryption function.}
% % 	\Question{Decrypt all files in the folder 'sample_data'. The folder is available on blackbaord. There are 100 files in the folder so turn the user verification off when de-encrypting the files. The files contain the names, customer numbers and number of adult, child and concession tickets for seven different movies. Store the contents of each file as the item of a data structure.}
% % \end{Exercise}

% % \begin{Exercise}[title=Extracting Information]
% % 	\Question{Use the }
% % 	\Question{}
% % 	\Question{}
% % \end{Exercise}

% \vspace{1em}
% \hrule
% \vspace{1em}


\begin{Exercise}[title = Enhancing your Program]

Implement your own idea to enhance your Caesar cipher program further. 

You \textbf{must} comment all enhancements using the following format to enable allow the person marking your project to easily locate your enhancements.:

\vspace{0.5em}
{\tt \#!EXTRA\# Comment here ...}
\vspace{0.5em}

An example comment:

\vspace{0.5em}
{\tt \#!EXTRA\# Generate bar chart showing frequency of the ten most common words}
\vspace{0.5em}

Suggested enhancements:
\begin{itemize}
    \item Produce a bar chart showing (up to) the ten most common words (Part \ref{Ex:Analysing_Msgs})  on the horizontal axis and the number of times they appear in the message on the vertical axis.
    \item Save the plot as a .pdf file.
    \item Create a Caesar cipher Python module that is imported to perform encryption and decryption operations within a main program. 
    \item Write a Caesar cipher class containing encryption and decryption methods.
    \item Implementing a different type of cipher in Python \\ e.g. \url{https://www.tutorialspoint.com/cryptography/traditional\_ciphers.htm}
    
    
\end{itemize}
\end{Exercise}

\vspace{1em}
\hrule
\vspace{1em}

% \subsection*{Helpful hints and suggestions}
% \begin{itemize}
%     \item It is crucial that you complete all of the exercise sheets first - they have been specifically designed to prepare you for this assignment.
% 	\item You should spend plenty of time planning out how the flow of your program might look \textbf{before} you start programming. Look back at previous exercise sheets to see how to approach larger problems by breaking them down into smaller ones.
% 	\item Remember to think about the \textbf{readability} and \textbf{reusability} of your code. Some question you might ask yourself:
% 	\begin{itemize}
% 		\item Have I named things sensibly? Could someone pick up my code and understand it?
% 		\item Am I repeating lots of code? Can I reuse any?
% 		\item Can I simplify the layout of my code to make it more readable? 
% 	\end{itemize}
% 	\item \textbf{Comment your code!} Comments are crucial for many reasons:
% 	\begin{itemize}
% 		\item They help other readers to understand what your code is doing.
% 		\item They help \emph{you} remember what your code is doing, when you come back to it weeks/months/years later. The whole point of writing code is that it can be reused many times, so to make it reusable, add comments.
% 		\item To help your TA assess your understanding of the problem you are trying to solve, and to assess how well your solution solves it. If you use a data structure to implement some functionality, explain it in the comments.
% 	\end{itemize}
% 	Short comments are great to summarise the next several lines of code by giving a high-level overview. For longer, more detailed comments (likely useful in functions and for Part 4), by enclosing them in three speech marks {\tt """} either side. E.g.
	
% 	\vspace{0.5em}
% 	{\tt def SomeFunction(SomeArgument):}\\
% 	{\tt \hspace*{2em}"""}\\
% 	{\tt \hspace*{2em}This function does ...}\\
% 	{\tt \hspace*{2em}....and it does this....}\\
% 	{\tt \hspace*{2em}"""}
% 	\vspace{0.5em}
	
% 	\item We mark for the following points:
% 	\begin{itemize}
% 	    \item Functionality: The more of the functionality described above you implement, the higher the mark.
% 	    \item Types: Proper use of data types and data structures. 
% 	    \item Comments: Appropriate and useful comments. 
% 	    \item Naming: Appropriate variable and function names (remember we want you to use CamelCase)
% 	    \item Report: Informative report, which explains your thought process and analyses the choices you made in your code design. You must reference any external code or ideas used.
% 	    \item Working code: Does the code do what it is supposed to? {\bf READ THE QUESTIONS CAREFULLY to check}. Is the program robust, i.e., does it deal correctly with wrong inputs (user/program interaction).
% 	\end{itemize}
	
% 	\item Finally, don't forget to ask for help! You will be able to ask your TA for help and advice during the weekly drop-in sessions and tutorials. 
% \end{itemize}



% \begin{Exercise}[title=Using Data From imported files]
%     \Question{Use your program to decrypt the file {\tt douglas\_data\_encrypted.txt}, a data set about wooden beams.}
%     \Question{Plot the {\bf knot ratio}  against the {\bf bending strength}  of the beams. \newline Fit a trend-line to the data and display this on the same plot along with the equation of the fitted line.}
%     \Question{The root mean square error (RMSE) is metric often used to judge how well a trend line fits the data. The RMSE is the square root of the mean of the sum of the error, $\varepsilon_i$, squared, for all data points. A smaller RMSE indicates a better fit between raw and fitted data.  $$RMSE=\sqrt{\frac{1}{N}\sum_{i=1}^{N}{\varepsilon_i^2}}$$
%     The error $\varepsilon_i$ is the difference between the raw data point, $a$, and the fitted data point, $y$, for some input value $x_i$:
% $$\varepsilon_i = a(x_i) - y(x_i)$$ Find the RMSE for the raw data and fitted data on the knot ratio vs. beam strength plot and display the value on the plot.}
%     \Question{Save the plot as a .pdf file.}
%     \Question{The bending stiffness $k$ of a cantilever beam can be found by: $$ k = \frac{3EI}{L^3} $$ where $E$ is the Young's modulus, $L$ is the distance from the beam support, $I = \frac{b^4}{12}$ is the area moment of inertia of a beam with a square cross section with sides of length $b$ (Figure 1).%} 
    
%     \vspace{5mm}
    
%     % \begin{figure}[t]
%     % \begin{center}
%     % \includegraphics[width=7cm]{cantilever.png}
%     % \caption{Parameters of a square beam configured as a cantilever}
%     % \label{cantilever}
%     % \end{center}
%     % \end{figure}
%     %}
    
%     %\makebox[0pt][l]{%
%     \begin{minipage}{\textwidth}
%     %\centering
%     \begin{center}
%     \includegraphics[width=.5\textwidth]{cantilever.png}
%     %\caption{Figure 1: }{figure caption}
%     \label{fig:fig1}
%     \end{center}
%     \end{minipage}
    
%     \vspace{5mm}
    
%     \begin{center}
%     \caption{Figure 1: Beam configured as a cantilever showing parameters to calculate bending stiffness}
%     \end{center}
    
%     \vspace{5mm}
%     }
    
    
%     Modify your program to:
%     \begin{enumerate}[label=(\alph*)]
% 		\item Use parameters {\tt E} and {\tt BEAMHEIGHT} from the decrypted file to find the bending stiffness in Nmm$^{-1}$ for each beam. Assume the beam is configured as a cantilever where $L = $ ({\tt beam height} - $d$), $d = 10$ cm. {\bf Take care to check the units used in the decrypted file!}
% 		\item Save the data-set as a .csv file:
% 		\begin{itemize}
% 		    \item Include the bending stiffness as an extra column.
% 		    \item Exclude the {\tt SAMPLE} column from the original data-set.
% 		\end{itemize} 
% 	\end{enumerate}
%     }

% \end{Exercise}

% \begin{Exercise}[title=Importing mulitple files]
% 	\Question{Modify your program so the user now has an additional choice of importing a directory of files in addition to providing an input message and reading a message from a file. Like when importing a file, the program should first check if the directory exists and raise an error of it is not found that prompts the user to enter the name correctly.}
% 	\Question{If the option to import directory is chosen, the program should read in directory, and decrypt/encrypt the contents of each file.}
% 	\Question{Modify the auto-decryption function so that the input from the user to verify if the decryption was successful can be switched on/off using an input argument to the auto-decryption function.}
% 	\Question{Decrypt all files in the folder 'sample_data'. The folder is available on blackbaord. There are 100 files in the folder so turn the user verification off when de-encrypting the files. The files contain the names, customer numbers and number of adult, child and concession tickets for seven different movies. Store the contents of each file as the item of a data structure.}
% \end{Exercise}

% \begin{Exercise}[title=Extracting Information]
% 	\Question{Use the }
% 	\Question{}
% 	\Question{}
% \end{Exercise}





\end{document}
