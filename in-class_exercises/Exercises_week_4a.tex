%%%%%%%%%%%%%%%%%%%%%%%%%%%%%%%%%%%%%%%%%%%%%%
% Header
\documentclass[11pt]{report}
\usepackage{amsmath}
\usepackage[english]{babel}
\usepackage[utf8x]{inputenc}
\PassOptionsToPackage{hyphens}{url}\usepackage{hyperref}
\usepackage{graphicx}
\usepackage{fullpage}
\usepackage{nicefrac}
\usepackage[lastexercise]{exercise}
\usepackage[dvipsnames]{xcolor}
\usepackage{listings}

\setlength{\parindent}{0cm}

\renewcommand{\ExerciseHeader}{\large\textbf{\ExerciseName~\ExerciseHeaderNB} - \textbf{\ExerciseTitle}\medskip}

\renewcommand{\ExePartHeader}{\medskip\textbf{\ExePartName\ExePartHeaderNB\ExePartHeaderTitle\medskip}}

\begin{document}
%%%%%%%%%%%%%%%%%%%%%%%%%%%%%%%%%%%%%%%%%%%%%%
\subsubsection*{EMAT10007 -- Introduction to Computer Programming}
\subsection*{\Large Exercises -- Week 4. Functions}

To complete this week's exercises you will also need to download the {\tt Birthday.py} file from Blackboard.


There was some confusion last week on the fill in the blank ({\tt <?>}) questions. The exercise sheets use {\tt <?>} as a placeholder to represent something missing - this could be an operator, a variable name, function name, etc. Multiple {\tt <?>} in the same question are not necessarily representing the same thing.


\begin{Exercise}[title=Happy Birthday]
    \Question{Open the {\tt Birthday.py} file and run it.}
    \Question{Add the line
    
    \vspace{1em}
    {\tt def BirthdaySong():} 
    \vspace{1em}
    
    to create a function.}
    \Question{What happens when we run this now? Remember we need to call a function for it to run. Add a line which calls the function.}
    \Question{How would you change the function so we now celebrate Lynn's birthday?}
    \Question{Add the argument {\tt Name} to the function and rewrite the function so we can celebrate anyone's birthday. Call it to celebrate Dan's birthday.}
    \Question{Note that we repeat the line
    
    \vspace{1em}
    {\tt print("Happy Birthday to you!")} 
    \vspace{1em}
    
    multiple times. One of the benefits of functions is that they reduce the amount of repeating code.
    
    Create a function called {\tt HappyBirthday} which prints the phrase {\tt "Happy Birthday to you!"}. 
    }
    \Question{Insert this new function {\tt HappyBirthday} into the original {\tt BirthdaySong} function.}
    \Question{Now this has not really made a big difference to the repetitions in the code.
    
    Change the {\tt HappyBirthday} function to take a argument {\tt Repeats} with a default value of 1. It should then print the phrase the given number of times.}
    \Question{Update the {\tt BirthdaySong} function with the improved {\tt HappyBirthday} function - eliminating as much repeated code as possible.}
\end{Exercise}

\begin{Exercise}[title=Powers]
    \Question{Can you complete the {\tt RetSquared} function below by replacing the blank {\tt <?>} with the correct variable name?
    
    \vspace{1em}
    {\tt def RetSquared(Number):}\\
    {\tt \hspace*{2em}\# Returns the square}\\
    {\tt \hspace*{2em}Squared = <?> ** 2}\\
    {\tt \hspace*{2em}return(<?>)}
    \vspace{1em}
    }
    \Question{Use the {\tt RetSquared} function and a {\tt for} loop to print the squares of all integers from 1 to 10.}
    \Question{Create a new function called {\tt RetCubed} and use it to print cubes of all integers from 1 to 10.}
    \vspace{-\baselineskip}
    \Question{Can you write a general function for raising numbers to powers? The function should take two arguments {\tt Number} and {\tt Power} and return the number raised to the given power? Remember to give your new function a sensible name.}
    \Question{Use your new general function to print the powers of 2 up to $2^{10}$.}
    \Question{Change this function so that the argument {\tt Power} has the default value {\tt 1}.}
    \Question{Write a new function which has arguments {\tt Number} and {\tt Powers}. If I have inputs {\tt Number=2} and {\tt Powers=[1, 3, 4]}, the function should return the list {\tt [2, 8, 16]}.
    
    How else can you write a function when you don't know the exact number of inputs?}
    \Question{(*) Can you write a function which returns the prime factorization of a number?
    
    \textbf{Note:} Learn about prime factorization here: \url{https://www.mathsisfun.com/prime-factorization.html}}
\end{Exercise}

\begin{Exercise}[title=Supermarkets] \label{Ex:Supermarkets}

    You have been put in charge of stock and pricing for a local greengrocer. They present you with the following table of items, stock, cost and price:
    \begin{center}
     \begin{tabular}{||c c c c||} 
     \hline
     Item & Stock & Cost & Price \\ [0.5ex] 
     \hline\hline
     Apple & 35 & 0.1 & 0.2 \\ 
     \hline
     Avocado & 10 & 0.7 & 1 \\
     \hline
     Bananas & 0 & 0.08 & 0.15 \\
     \hline
     Gai Lan & 10 & 0.75 & 1.3 \\
     \hline
     Carrots & 15 & 0.05 & 0.1 \\
     \hline
     Pomelo & 2 & 1.2 & 1.5 \\
     \hline
     Tomatoes & 20 & 0.35 & 0.50 \\
     \hline
     Pineapple & 0 & 0.8 & 1.25 \\ [1ex] 
     \hline
    \end{tabular}
    \end{center}
    \Question{Using the information in the table create three dictionaries {\tt ItemStock}, {\tt ItemCost} and {\tt ItemPrice}.}
    \Question{Write a function which returns the lowest cost item.}
    \Question{Write a function which returns the price for any specified item. Make sure to let the user know if the item isn't sold (i.e. it is not in the dictionary).}
    \Question{Write a function which returns a list of items which have completely sold out (i.e. stock is 0).}
    \Question{Write a function which returns a list of items which are below a given stock level. How would you set it so the default threshold for low stock was 5?}
    \Question{Write a function which returns the profit margin for a given item (i.e. Price - Cost)}
    \Question{Write a function which prints out the prices of all the items in stock like so:
    
    \vspace{1em}
    {\tt Apple : 20p}\\
    {\tt Avocado : £1}
    \vspace{1em}
    }
    \Question{Write a function which reads in a number of items from a user using the {\tt input} function and returns the total price of all those items. How would you extend this to also tell the user if one of their item is unavailable? 
    
    \textbf{Hint:} Can you use an answer to a previous question here?}
    \Question{Don't forget if we sell an item then we need to reduce the stock count for that requested item. Add this to your previous function.}
    \Question{Suppose the user now wants to add new items to the inventory. Write functions which read in the item, stock, cost and price from the user using the {\tt input} function and then update the dictionaries respectively.
    
    \textbf{Note:} Think about the re-usability of your code here and the merits of having separate functions for updating each dictionary.}
    \Question{Finally, invite your TA to do some shopping!}
    \Question{(*) It is slightly clunky having three different dictionaries here. How would you improve upon this?}
\end{Exercise}

\begin{Exercise}[title=Word Scrambler]

    An old internet meme once claimed that if you scrambled the words of a sentence in such a way that the first and last letters remained in place, but the rest of the letters were shuffled, it could still be understood. The following sentence was used to promote the  claim:\\
	
	\emph{``Aoccdrnig to a rscheearch at Cmabrigde Uinervtisy, it deosn't mttaer in waht oredr the ltteers in a wrod are, the olny iprmoetnt tihng is taht the frist and lsat ltteer be at the rghit pclae. The rset can be a toatl mses and you can sitll raed it wouthit porbelm. Tihs is bcuseae the huamn mnid deos not raed ervey lteter by istlef, but the wrod as a wlohe.''}\\
	
	It was later proven incorrect; however, it makes for an interesting programming exercise! \\
	
	How would we write a program which reads in a phrase and prints out the scrambled words? Let's break it down into steps.\\
	
    \Question{First let's just consider how would we completely scramble a single word? For example, {\tt "scramble"} could become {\tt "lbeacmrs"} or {\tt "toblerone"} could become {\tt "loonberet"}.
    
    \textbf{Hint:} Recall the {\tt shuffle} function in the {\tt random} module. However, be aware that {\tt shuffle} rearranges elements in place and so cannot work on strings - you will need to convert to a list.
    }
    \Question{So now we can scramble the entire word - how would you adapt this to keep the first and last letters in place? For example, {\tt "scramble"} could become {\tt "smlabcre"} or {\tt "toblerone"} could become {\tt "tnobleroe"}.}
    \Question{Now rewrite this as a function called {\tt WordScrambler} which takes in an argument {\tt Word}.}
    \Question{Almost there - can you apply your function to every word in a list of words? For example, the list {\tt ["scramble", "toblerone", "smell", "cheese"]}?}
    \Question{Finally, rather than have a list of words it'd be nicer to read in a sentence from the user using {\tt input()} and print out the scrambled sentence. How would you do this?
    }
    \Question{(*) Enhance your word scrambler. Some ideas include:
    \begin{itemize}
        \item Leave punctuation and numbers unchanged.
        \item Add a flag to the function which switches functionality between scrambling the entire word and leaving the first and last letters in place.
        \item How would you create a scrambler which only scrambles the position of every other letter?
    \end{itemize}}
\end{Exercise}

\begin{Exercise}[title=Fibonacci Sequence]

The Fibonacci sequence is defined in the following way:
\begin{equation*}
    F_0 = 0, \quad F_1=1,
\end{equation*}
and
\begin{equation*}
    F_n = F_{n-1} + F_{n-2},
\end{equation*}
where $F_n$ is the $n^{th}$ term in the sequence.
    \Question{Create a function which calculates the next term in the Fibonacci sequence given the previous two.}
    \Question{Now create a function {\tt Fib\_iterative} which calculates the Fibonacci sequence up to a given limit iteratively.}
    \Question{Can you now write alternative function {\tt Fib\_recursive} which works recursively?}
    \Question{Compare the advantages and disadvantages of the two approaches. Which one is easier to implement? Which one works better for large $n$?}
    \Question{(*) Can you research another way to calculate Fibonacci numbers which does not require you to know the previous two? What are the advantages and disadvantages of this approach?}
    \Question{(*) How would you calculate the results for $n<0$?}
\end{Exercise}

\begin{Exercise}[title=(*) Additional Problem Ideas]

    If you finish all the exercises in this week's sheet, here are some additional problems:
    \Question{Look back through previous exercise sheets and think about how you would solve them now using functions. Try some! I recommend:
    \begin{itemize}
        \item The circles question at the end of Week 2 - Exercise 7
        \item FizzBuzz Game from Week 3 - Exercise 8
    \end{itemize}}
    \Question{Exercise \ref{Ex:Supermarkets} showed how dictionaries can be used to store information like databases and then queried using functions. Can you think of some other useful applications for this? For example, maybe a record of the books you have read and your review scores for them?}
    \Question{Explore different solutions for the Tower of Hanoi problem - description here: \url{https://en.wikipedia.org/wiki/Tower_of_Hanoi}}
    \Question{Hangman - you can find the rules here: \url{https://m.wikihow.com/Play-Hangman}}
    \Question{Blackjack - you can find the rules here: \url{https://entertainment.howstuffworks.com/how-to-play-blackjack.htm}}
\end{Exercise}
\end{document}