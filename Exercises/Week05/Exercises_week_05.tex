%%%%%%%%%%%%%%%%%%%%%%%%%%%%%%%%%%%%%%%%%%%%%%
% Header
\documentclass[11pt]{report}
\usepackage[english]{babel}
\usepackage[utf8x]{inputenc}
\PassOptionsToPackage{hyphens}{url}\usepackage{hyperref}
\usepackage{graphicx}
\usepackage{fullpage}
\usepackage{nicefrac}
\usepackage[lastexercise]{exercise}
\usepackage[dvipsnames]{xcolor}
\usepackage{amsmath}


\usepackage{minted}
\makeatletter
\AtBeginEnvironment{minted}{\dontdofcolorbox}
\def\dontdofcolorbox{\renewcommand\fcolorbox[4][]{##4}}
\makeatother


\setlength{\parindent}{0cm}

\renewcommand{\ExerciseHeader}{\large\textbf{\ExerciseName~\ExerciseHeaderNB} - \textbf{\ExerciseTitle}\medskip}

\renewcommand{\ExePartHeader}{\medskip\textbf{\ExePartName\ExePartHeaderNB\ExePartHeaderTitle\medskip}}

\begin{document}

%%%%%%%%%%%%%%%%%%%%%%%%%%%%%%%%%%%%%%%%%%%%%
\subsubsection*{Introduction to Computer Programming}
\subsection*{\Large Exercises -- Week 5. Classes}

\subsection*{\Large Part 1. Classes in Python}

\begin{Exercise}[title=Defining classes (Essential)]

  The goal of this exercise is to write a program that creates a shopping list and then prints out all of the items and the total price.
  
  \Question{Write a class called {\tt Item} that has a constructor
    \begin{minted}{python}
      __init__(self)
    \end{minted}
    that prints ``This is an item". From your main program, create an object of class Item called Apple. Run the program.}
  
  \Question{Change the constructor to include three additional arguments:

    \begin{minted}{python}
      __init__(self, Description, Number, UnitPrice)
    \end{minted}

    Use these arguments to create three new class attributes.
    
    }
  
  \Question{Change the print statement to print ``Created a new item: {\tt X}'', where {\tt X} is the item description.}

  \Question{From your main program, create an object of class Item called Apple with parameters ``Apple", 1, and 0.5. Call {\tt print(Apple.Description,Apple.Number,Apple.UnitPrice)} to print out the information.}
  
  \Question{ Include a function in your class called {\tt PrintItemInfo(self)} that prints all the information about the item. Call {\tt Apple.PrintItemInfo()} from the main program.}
  
  \Question{Override the built-in \verb|__str__()| function so that printing the instance of {\tt item} prints the name of that item. Then, create a list called {\tt ShoppingList}, add the Apple, and two other items to the list. Loop through the list and print out each item using the overridden \verb|__str__()| method.}
  
  \Question{ Write a loop in the main program to go over all items in {\tt ShoppingList}, print out the item information and sum the total price (the price for one item is {\tt Number*UnitPrice}). Print out the total price at the end.}

\end{Exercise}
  
%%%%%%%%%%%%%%%%%%%%%%%%%%%%%%%%%%%%%%%%%%%%%%%%%%%%%%%%%%%%%%%%%%
  
\subsection*{\Large Part 2. Inheritance}

\begin{Exercise}[title=Deriving a class with inheritance (Essential)]

\Question{Add a new class called {\tt SpecialItem}
  which \emph{inherits} from the {\tt Item} class.
  The class signature should look like the following:
  \begin{minted}{python}
    def __init__(self, Description, Number, UnitPrice, SpecialInfo):
  \end{minted}
  and should call the \verb|__init__()| function of the Item class passing in the {\tt Description}, {\tt Number}, and {\tt UnitPrice} arguments, but storing the new variable {\tt SpecialInfo} as an attribute of the new class.}

\Question{As we did in the {\tt Item} class, override the built in \verb|__str__()| function to print the item description, but this time have it also print out the special information via {\tt self.SpecialInfo}.}

\Question{Override the {\tt PrintItemInfo()} method of the {\tt Item} class
  so that the special information is also printed.}

\Question{Add a special item to your shopping list that requires instructions via the {\tt SpecialInfo} argument, such as Paracetamol, which has the following special information: take two tablets every 6 hours. Check that {\tt print(Paracetamol)} and {\tt Paracetamol.PrintItemInfo()} work as expected. Verify that
  {\tt Apple.PrintItemInfo()} works the same as before.}
\end{Exercise}


\subsection*{\Large Part 3. Advanced questions}

\begin{Exercise}[title=A class for vectors]

  The purpose of this exercise is to create a class for vectors and
  carrying out operations on vectors. We'll consider vectors of
  the form $\vec{v} = \langle x, y, z \rangle$.

  \Question{For this exercise, we'll need some additional mathematical
    functions that are not available in Python by default. To enable these,
    add the line 
    \begin{minted}{python}
      from math import *
    \end{minted}
    You should now be able to compute square roots using the {\tt sqrt}
    function. Trigonometric functions (sin, cos, tan, etc) will be
    available now too.}

  \Question{Create a class called {\tt Vector} with attributes
    {\tt x}, {\tt y}, {\tt z}.}

  \Question{Overwrite the \verb|__str__()| function so that the {\tt print()}
    function can be used to print the vector in the form {\tt <x, y, z>}.
    Check that this works by creating a vector $\vec{v} = \langle 1, 3, 2 \rangle$ and then calling {\tt print(v)}.}

  \Question{Add a function call {\tt norm} that computes the length 
    of a vector, defined as $|\vec{v}| = \sqrt{x^2 + y^2 + z^2}$. Compute the 
    length of the vector $\vec{v}$.}

  \Question{Overload the + operator by defining the \verb|__add()__| function
    in the vector class.
    Remember that addition is a binary operation, so the \verb|__add__()|
    function requires two arguments: {\tt self} and {\tt other}. 
    Define a second vector $\vec{w} = \langle 5, 0, 1 \rangle$ and
    check that $\vec{v} + \vec{w} = \langle 6, 3, 3 \rangle$.}

  \Question{Now we'll overload the * operator so that it computes the
    dot product of two vectors. Recall that if $\vec{v} = \langle x,y,z \rangle$
    and $\vec{w} = \langle a, b, c \rangle$ then 
    $\vec{v}.\vec{w} = ax + by + cz$. The * operator can be overloaded by
    defining the
    \verb|__mul__()| function in the vector class.}

  \Question{Use these methods and operations to compute the angle between
    the vectors $\vec{v}$ and $\vec{w}$. Recall that the angle $\theta$
    between two vectors is defined by
    \begin{align*}
      \theta = \arccos\left(\frac{\vec{v}.\vec{w}}{|\vec{v}| |\vec{w}|}\right).
    \end{align*}
    The function $\arccos$ is defined in Python as {\tt acos}.}

  \Question{(Very advanced) Let's suppose that we want to multiply
    a vector $\vec{v} = \langle x, y, z \rangle$ by a float $f$
    such that $f \vec{v} = \vec{v} f = \langle fx, fy, fz \rangle$. This can also be done
    by overloading the * operator, but there are some subtleties. One issue
    is that we have already defined * using the \verb|__mul__()| function
    in Question 6 and this assumes the * operation is being applied to
    two vectors. To overcome this issue, redefine the \verb|__mul__()|
    function and use an {\tt if} statement to determine which operation to
    carry out based on the type of {\tt other}. Check that this works by
    computing {\tt v * w} and {\tt v * 1.0}.
    \\[1em]
    Now try to compute {\tt 1.0 * v}. You'll notice an error occurs.
    This is because the order of arguments matters when calling Python
    functions. Writing {\tt v * 1.0} is the same as calling
    \verb|v.__mul__(1.0)|. Since we define the function \verb|__mul__|
    in the vector class, we can provide instructions for how to
    evaluate this function when the argument is a float. However,
    writing {\tt 1.0 * v} calls the \verb|__mul__| function defined
    in the float class, and Python doesn't know how to evaluate this
    function when it is passed a vector. Thankfully, there is an easy
    fix which avoids editing the float class. This involves defining
    the reflected multiplication function \verb|__rmul__|
    in the vector class, which has two arguments {\tt self} and {\tt other}.
    When the command {\tt 1.0 * v} is executed, Python first tries to call
    \verb|1.0.__mul__(v)| and when this fails, it will then try to run
    \verb|v.__rmul__(1.0)|. This means that the reflected multiplication
    function can be defined in the same way as the normal multiplication
    function. Implement the \verb|__rmul__| function in your vector class
    and verify that it works by calculating {\tt 1.0 * v}. 
    }
    

\end{Exercise}

\end{document}