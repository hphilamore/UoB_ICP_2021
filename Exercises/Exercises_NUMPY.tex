%%%%%%%%%%%%%%%%%%%%%%%%%%%%%%%%%%%%%%%%%%%%%%
% Header
\documentclass[11pt]{report}
\usepackage[english]{babel}
\usepackage[utf8x]{inputenc}
\PassOptionsToPackage{hyphens}{url}\usepackage{hyperref}
\usepackage{graphicx}
\usepackage{fullpage}
\usepackage{nicefrac}
\usepackage[lastexercise]{exercise}
\usepackage[dvipsnames]{xcolor}
\usepackage{listings}
\usepackage{enumitem}
\graphicspath{ {./img/} }

\begin{document}

\setlength{\parindent}{0cm}

\renewcommand{\ExerciseHeader}{\large\textbf{\ExerciseName~\ExerciseHeaderNB} - \textbf{\ExerciseTitle}\medskip}

\renewcommand{\ExePartHeader}{\medskip\textbf{\ExePartName\ExePartHeaderNB\ExePartHeaderTitle\medskip}}
%%%%%%%%%%%%%%%%%%%%%%%%%%%%%%%%%%%%%%%%%%%%%%
\title{Exercises -- Week 7: Numpy}
\subsubsection*{EMAT10007 -- Introduction to Computer Programming}
\subsection*{\Large Exercises -- Week 7. Numpy}

Exercises this week will explore the use of Numpy, a python module designed for efficient and straightforward manipulation of arrays.
\\


\subsection*{\Large Part 1. User input}
\begin{Exercise}[title= Element-wise operations] \label{Ex:Variables}
	\Question{ Create a list of integer values from 2 to 9 called {\tt numsList}. Display the list using the {\tt print} function.}
	\Question{Now convert that list to a numpy array called {\tt numsArray}. Display the array by using {\tt print} again. Does this differ in any way from the list display?}
	\Question{Add 3 to each element in {\tt numsList} using list comprehension. Then add 3 to each element in {\tt numsArray}.
	Check your outputs are the same using {\tt print}. 
	}
	\Question{Subtract 2 from each element in {\tt numsList} and {\tt numsArray}.
	Check your outputs are the same. 
	}
	\Question{Create a new variable called {\tt x} in a single line of code, which is a numpy array containing numbers from 1 to 10. \\
	\textbf{Hint:} Use {\tt linspace}.}
	\Question{Check the type of each element in {\tt x}. Can you think of different ways of creating this numpy array so that each element is of the type: {\tt numpy.int64}? }
    \Question{Multiply each element in {\tt x} by 2, then print the first 5 elements of {\tt x}. Your output should look like this:\\ {\tt [2. 4. 6. 8. 10.]}}
    \Question{Finally, divide each element of {\tt x} by 2, and print out only the even elements. Your output should look the same as for the previous question.}
\end{Exercise}

\begin{Exercise}[title= Shaping arrays] \label{Ex:Variables}
	\Question{ Create a 2d numpy array named {\tt y} that looks like the following :\\
	{\tt [[2. 2. 2.],}\\
	   {\tt [2. 2. 2.]]}\\
	    \textbf{Hint:} Use the numpy {\tt ones()} function.}
	\Question{Modify {\tt y} so it has the following shape:\\
	{\tt [[2. 2.],}\\
	{\tt[2. 2.],}\\
	   {\tt [2. 2.]]}\\
    \textbf{Hint:} Use the numpy {\tt reshape()} function.}
    \Question{What happens if you now try {\tt y = y.reshape(4,2)? Why?}
    \Question{Modify {\tt y} to look like the following: \\
    {\tt [[1. 2.],}\\
    {\tt [3. 4.],}\\
    {\tt [5. 6.]]}\\
    
    \Question{Perform an element-wise multiplication of {\tt y} with itself and save the output as a new array {\tt z}}.
    \Question{Use the numpy {\tt dot} function to perform a dot product of {\tt y} with itself. Remember you may need to reshape the array for this to work, refer to the lecture slides if needed.\\
   \textbf{Note:} There is a specific type of reshape that is called a transpose. Try printing the output of {\tt y.T} }
    
\end{Exercise}

\subsection*{\Large Part 2. Numpy Functionality}

\begin{Exercise}[title= Lottery game]

\ExeText{Here we will code a short program to implement a "Lotto" style game, in which 3 numbers from 1-20 are chosen by the user and then randomly drawn by our program. }

	\Question{Use the {\tt input} function to have the user input a list of 3 numbers from 1-20, and convert this to a numpy array named {\tt guessedNumbers}.}
	\Question{Implement additional checks in the input phase to ensure the user inputs an array of 3 numbers, with each number being in the range 1-20. If an incorrect input is entered by the user, ask them to re-enter the input until it is in the correct format.}
	\Question{Use the numpy {\tt random} function to produce an array of random numbers from 1 to 20, named {\tt lotteryNumbers}.}
	\Question{Check whether the arrays are equivalent, to see if the user won the jackpot. If they get all three numbers right, print out the following: \\
	{\tt Congratulations! You guessed all 3 numbers and won the jackpot!}\\
	\textbf{Hint:} You could sort the arrays and compare them using the {\tt ==} operator.}
    \Question{Numbers can currently be repeated both in {\tt guessedNumbers} and {\tt lotteryNumbers}. Implement additional checks on the user input to avoid this. Then, ensure the randomly generated numbers are non-repeated. \\
    \textbf{Hint:} For {\tt lotteryNumbers}, try modifying the following function to fit your needs:\\
    {\tt np.random.choice(range(20), 10, replace=False)}} 
	\Question{(*) Your program only checks for the jackpot. Think of ways to extend its functionality so it can tell the user how many numbers they guessed right.}

\end{Exercise}


\end{document}